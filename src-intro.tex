In quantum field theories a common problem arising is the presence of divergences that have to be taken care of by regularization and renormalization. There are several kinds of divergences that are distinguished, in particular infrared (IR) and ultraviolet (UV). The former appear in theories with massless particles and stem from contributions of objects with energy tending to zero. They can be dealt with, e.g., by imposing an infrared cutoff\footnote{that is setting a lower limit for the energy} or assigning the massless particle a fictitious mass and removing the cutoff or the mass later by taking the limit. In contrast, UV divergences stem from contributions of objects with energy tending to infinity. There are various techniques of renormalization procedures for different UV problems. The one that is relevant in the present case is briefly outlined later. Even though the renormalized theories yield physically sensible results, it might give different or even deeper insight into the problem to have a formulation for the theory that does not exhibit the need for renormalization.
\\
In case of the UV problem that appears here for the annihilation and creation of particles at point sources, such a formulation was proposed by Teufel and Tumulka in \cite{56e26bd5} and \cite{f5e1d34b}. It makes use of so called interior-boundary conditions (IBCs) that couple sectors of Fock space with different numbers of particles. The formulation is given in the particle-position representation for vectors in Fock space. This means such a vector is represented by a wavefunction on a configuration space with a variable number of particles\footnote{This configuration space is given by the disjoint union $\bigcup_{n = 0}^{\infty}\mathbb{R}^{dn}$, where $d$ is the considered dimension.}. The absorption of a particle $1$ by a particle $2$ is then given by a jump from the configuration where both particles are at the same location to a configuration without particle $1$. The emission of a particle is given by the opposite jump. Regarding the configurations where two particles are at the same location as the boundary of the configuration space, such a jump links a boundary point of one sector of Fock space to an interior point of another sector and can be given by an IBC (hence the name). Suitable wave functions, that satisfy the IBC, then allow for a flux of probability into (or out of) the boundary so that the probability is not conserved sector-wise but still on the whole Fock space. This probability flux between sectors takes account of the possibility of particle annihilation and creation. These wave functions, however, are not in the domain of the free field Hamilton operator, so that a different domain has to be chosen for the Hamilton operator of the IBC-model.
\\
For the case of non-relativistic models for scalar particles with fixed sources in three spatial dimensions Lampart, Schmidt, Teufel and Tumulka gave mathematically rigorous results in \cite{7f9f7f95}. They prove that the relevant IBC Hamilton operator is self-adjoint and bounded from below. The models discussed there are known to be renormalizable when described with methods of conventional quantum field theory and in the paper it is also shown that the renormalized Hamilton operator is equal to the IBC-version up to a finite additive constant, i.e. a shift in energy. The case of dynamical sources, i.e. moving particles at whose positions particles of a different kind can be annihilated and created, is not covered in that paper, but was described in \cite{f5e1d34b} already. A mathematically more detailed discussion is yet to come. As an interpretation for such a more realistic quantum field theory one could think of electrons that interact via photons.
\\
The present work describes the situation in two spatial dimensions. Its purpose is not to delve into all the mathematical details, but the approach given here rather follows the one presented by Keppeler and Sieber in \cite{3d622ca2}. In that paper they discuss models with IBCs in one spatial dimension. The main reason for not giving all mathematical details is that finding and dealing with the appropriate domains for the Hamilton operator and the other operators used here is not an easy task. The regularity properties of the functions that have to be considered can be quite complicated (compare \cite{7f9f7f95}). In return for the mathematical {\glqq}laxity{\grqq} the tasks can generally be discussed by explicit calculations.
\\
In section \ref{sec:genset} we first give the general setting and motivate the IBC Hamilton operator and the corresponding IBC. Section \ref{sec:grstate} is about the calculation of the ground state for the model. We first do this on a truncated Fock space with at most one particle existing. Afterwards we turn to the full Fock space. In section \ref{sec:sevsources} the case of more than one fixed source is considered. We start out with two sources at distinct positions and examine the dependence of the ground state energy on the distance of the sources. The situation is then generalized to the case of any finite number of fixed sources. The final section \ref{sec:symmetry} is about the symmetry of the Hamilton operator. There we give an explicit calculation assuming a sufficiently nice domain.
