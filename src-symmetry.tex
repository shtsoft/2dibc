The intention of this last section is to show the symmetry of the Hamilton operator $H$ from section \ref{subsec:grffsp} (i.e. we again assume the presence of only one source at the origin) on a certain subset $D \subset D_{\textrm{IBC}}$ by an explicit calculation, using integration by parts. In this way this not possible for all functions in $D_{\textrm{IBC}}$ as, e.g., applying the Laplace operator in only one variable does not necessarily yield a square-integrable function. To accomplish it anyway, one has to resort to different means than explicit calculations. Therefore, we choose $D$ to be the set of all functions in $D_{\textrm{IBC}}$ that are sufficiently regular, to guarantee that the derivatives and integrations taken in the following are well-defined. For the rest of this section let $\phi,\psi \in D$.
\\
Due to the linearity of the scalar product and since the terms of $H$ are individually well-defined on $D_{\textrm{IBC}}$, we can analyse the symmetry of the terms individually. Now because $E_{0} \in \mathbb{R}$, the multiplication with $E_{0}$ is obviously symmetric. Let therefore $H_{0}$ denote the Hamilton operator where $E_{0} = 0$. In the zero-particle sector we find
\begin{align*}
  \left(
      \phi^{0}
    \vert
      (H_{0}\psi)^{0}
  \right)_{0}
  &=
  \overline{\phi^{0}}
  (\overline{c}\mathcal{A}\psi)^{0}
  \\
  &=
  \overline{\phi^{0}}
  (\overline{c}\mathcal{A}\psi)^{0}
  -
  \overline{(\overline{c}\mathcal{A}\phi)^{0}}
  \psi^{0}
  +
  \left(
      (H_{0}\phi)^{0}
    \vert
      \psi^{0}
  \right)_{0}
  \,.
\end{align*}
For the $n$-particle sector with $n \geq 1$ we have
\begin{align*}
  &
  \left(
      \phi^{n}
    \vert
      (H_{0}\psi)^{n}
  \right)_{n}
  \\
  &=
  \int_{\mathbb{R}^{2n}}
  \overline{\phi^{n}}
  \left(
    x_{1}
    ,
    \dots
    ,
    x_{n}
  \right)
  \left(
    -
    \Delta^{\mathcal{F}}
    \psi
    +
    \overline{c}
    \mathcal{A}
    \psi
  \right)^{n}
  \left(
    x_{1}
    ,
    \dots
    ,
    x_{n}
  \right)
  \mathrm{d}^{2n}x
  \\
  &=
  -
  \sum_{j = 1}^{n}
  \int_{\mathbb{R}^{2n}}
  \overline{\phi^{n}}(x_{1},\dots,x_{n})
  \Delta_{j}
  \psi^{n}(x_{1},\dots,x_{n})
  \mathrm{d}^{2n}x
  \\
  &\phantom{=\ }
  -
  \int_{\mathbb{R}^{2n}}
  \overline{\phi^{n}}
  \left(
    x_{1}
    ,
    \dots
    ,
    x_{n}
  \right)
  \left(
    \frac{\overline{c}\sqrt{n + 1}}{2\pi}
    \lim_{r \downarrow 0}
    r
    \ln^{2}(r)
    \int_{S^{1}}
    \partial_{r}
    \frac{1}{\ln(r)}
    \psi^{n+1}(x_{1},\dots,x_{n},r\varphi)
    \mathrm{d}\varphi
  \right)
  \mathrm{d}^{2n}x
  \,.
\end{align*}
We use polar coordinates for the $j$-th argument and write $(\hat{x}_{j},r\varphi)$ for the whole argument then. Note that the angular part of the Laplace operator, for which we write $L_{j} = \left( \frac{1}{r^{2}}\Delta_{\varphi} \right)_{j}$, is symmetric for the considered $\phi,\psi$, as we assume that they are sufficiently regular away from $0$. To see this, one can simply parametrize in the integral over $S^{1}$ by
\begin{align*}
  \Phi
  \colon
  [0,2\pi]
  \to
  S^{1}
  ,\quad
  \theta
  \mapsto
  \Phi(\theta)
  &=
  (\cos(\theta),\sin(\theta))
  \,.
\end{align*}
Then the angular part of the Laplace operator is given by $\frac{1}{r^{2}}\partial_{\theta}^{2}$ and integrating by parts twice yields for $j \in [1,n] \cap \mathbb{N}$ that
\begin{align*}
  &
  -
  \int_{\mathbb{R}^{2n}}
  \overline{\phi^{n}}
  \left(
    x_{1}
    ,
    \dots
    ,
    x_{n}
  \right)
  L_{j}
  \psi^{n}
  \left(
    x_{1}
    ,
    \dots
    ,
    x_{n}
  \right)
  \mathrm{d}^{2n}x
  \\
  &=
  -
  \int_{\mathbb{R}^{2(n-1)}}
  \int_{0}^{\infty}
  \int_{0}^{2\pi}
  r
  \overline{\phi^{n}}(\hat{x}_{j},r\Phi(\theta))
  \frac{1}{r^{2}}
  \partial_{\theta}^{2}
  \psi^{n}(\hat{x}_{j},r\Phi(\theta))
  \mathrm{d}\theta
  \mathrm{d}r
  \mathrm{d}^{2(n-1)}\hat{x}_{j}
  \\
  &=
  -
  \int_{\mathbb{R}^{2(n-1)}}
  \int_{0}^{\infty}
  r
  \frac{1}{r^{2}}
  \left(
    \overline{\phi^{n}}(\hat{x}_{j},r\Phi(\theta))
    \partial_{\theta}
    \psi^{n}(\hat{x}_{j},r\Phi(\theta))
    \Big\vert_{0}^{2\pi}
  \right)
  \mathrm{d}r
  \mathrm{d}^{2(n-1)}\hat{x}_{j}
  \\
  &\phantom{=\ }
  +
  \int_{\mathbb{R}^{2(n-1)}}
  \int_{0}^{\infty}
  \int_{0}^{2\pi}
  r
  \frac{1}{r^{2}}
  \left(
    \partial_{\theta}
    \overline{\phi^{n}}(\hat{x}_{j},r\Phi(\theta))
  \right)
  \partial_{\theta}
  \psi^{n}(\hat{x}_{j},r\Phi(\theta))
  \mathrm{d}\theta
  \mathrm{d}r
  \mathrm{d}^{2(n-1)}\hat{x}_{j}
  \\
  &=
  -
  \int_{\mathbb{R}^{2(n-1)}}
  \int_{0}^{\infty}
  r
  \frac{1}{r^{2}}
  \left(
    \left(
      \partial_{\theta}
      \overline{\phi^{n}}(\hat{x}_{j},r\Phi(\theta))
    \right)
    \psi^{n}(\hat{x}_{j},r\Phi(\theta))
    \Big\vert_{0}^{2\pi}
  \right)
  \mathrm{d}r
  \mathrm{d}^{2(n-1)}\hat{x}_{j}
  \\
  &\phantom{=\ }
  -
  \int_{\mathbb{R}^{2(n-1)}}
  \int_{0}^{\infty}
  \int_{0}^{2\pi}
  r
  \frac{1}{r^{2}}
  \left(
    \partial_{\theta}^{2}
    \overline{\phi^{n}}(\hat{x}_{j},r\Phi(\theta))
  \right)
  \psi^{n}(\hat{x}_{j},r\Phi(\theta))
  \mathrm{d}\theta
  \mathrm{d}r
  \mathrm{d}^{2(n-1)}\hat{x}_{j}
  \\
  &=
  -
  \int_{\mathbb{R}^{2n}}
  \left(
    L_{j}
    \overline{\phi^{n}}(x_{1},\dots,x_{n})
  \right)
  \psi^{n}(x_{1},\dots,x_{n})
  \mathrm{d}^{2n}x
  \,,
\end{align*}
as $\Phi(0) = \Phi(2\pi)$ and $\partial_{\theta}\Phi(0) = \partial_{\theta}\Phi(2\pi)$.
\\
The interesting term is the term with the radial part of the Laplace operator. Remembering that $\phi^{n},\psi^{n}$ vanish sufficiently fast at infinity (i.e. the boundary terms where $r \to \infty$, vanish), we find for $j \in [1,n] \cap \mathbb{N}$ by integration by parts
\begin{align*}
  &
  -
  \int_{\mathbb{R}^{2n}}
  \overline{\phi^{n}}
  \left(
    x_{1}
    ,
    \dots
    ,
    x_{n}
  \right)
  (\Delta_{j} - L_{j})
  \psi^{n}
  \left(
    x_{1}
    ,
    \dots
    ,
    x_{n}
  \right)
  \mathrm{d}^{2n}x
  \\
  &=
  -
  \int_{\mathbb{R}^{2(n-1)}}
  \int_{S^{1}}
  \int_{0}^{\infty}
  r
  \overline{\phi^{n}}(\hat{x}_{j},r\varphi)
  \left(
    \partial_{r}^{2}
    +
    \frac{1}{r}
    \partial_{r}
  \right)
  \psi^{n}(\hat{x}_{j},r\varphi)
  \mathrm{d}r
  \mathrm{d}\varphi
  \mathrm{d}^{2(n-1)}\hat{x}_{j}
  \\
  &=
  -
  \int_{\mathbb{R}^{2(n-1)}}
  \int_{S^{1}}
  \int_{0}^{\infty}
  r
  \overline{\phi^{n}}(\hat{x}_{j},r\varphi)
  \partial_{r}^{2}
  \psi^{n}(\hat{x}_{j},r\varphi)
  \mathrm{d}r
  \mathrm{d}\varphi
  \mathrm{d}^{2(n-1)}\hat{x}_{j}
  \\
  &\phantom{=\ }
  -
  \int_{\mathbb{R}^{2(n-1)}}
  \int_{S^{1}}
  \int_{0}^{\infty}
  \overline{\phi^{n}}(\hat{x}_{j},r\varphi)
  \partial_{r}
  \psi^{n}(\hat{x}_{j},r\varphi)
  \mathrm{d}r
  \mathrm{d}\varphi
  \mathrm{d}^{2(n-1)}\hat{x}_{j}
  \\
  &=
  -
  \int_{\mathbb{R}^{2(n-1)}}
  \int_{S^{1}}
  \int_{0}^{\infty}
  r
  \overline{\phi^{n}}(\hat{x}_{j},r\varphi)
  \partial_{r}^{2}
  \psi^{n}(\hat{x}_{j},r\varphi)
  \mathrm{d}r
  \mathrm{d}\varphi
  \mathrm{d}^{2(n-1)}\hat{x}_{j}
  \\
  &\phantom{=\ }
  +
  \lim_{r \downarrow 0}
  \int_{\mathbb{R}^{2(n-1)}}
  \int_{S^{1}}
  \overline{\phi^{n}}(\hat{x}_{j},r\varphi)
  \psi^{n}(\hat{x}_{j},r\varphi)
  \mathrm{d}\varphi
  \mathrm{d}^{2(n-1)}\hat{x}_{j}
  \\
  &\phantom{=\ }
  +
  \int_{\mathbb{R}^{2(n-1)}}
  \int_{S^{1}}
  \int_{0}^{\infty}
  \left(
    \partial_{r}
    \overline{\phi^{n}}(\hat{x}_{j},r\varphi)
  \right)
  \psi^{n}(\hat{x}_{j},r\varphi)
  \mathrm{d}r
  \mathrm{d}\varphi
  \mathrm{d}^{2(n-1)}\hat{x}_{j}
  \\
  &=
  \lim_{r \downarrow 0}
  \int_{\mathbb{R}^{2(n-1)}}
  \int_{S^{1}}
  r
  \overline{\phi^{n}}(\hat{x}_{j},r\varphi)
  \partial_{r}
  \psi^{n}(\hat{x}_{j},r\varphi)
  \mathrm{d}\varphi
  \mathrm{d}^{2(n-1)}\hat{x}_{j}
  \\
  &\phantom{=\ }
  +
  \int_{\mathbb{R}^{2(n-1)}}
  \int_{S^{1}}
  \int_{0}^{\infty}
  \left(
    \partial_{r}
    r
    \overline{\phi^{n}}(\hat{x}_{j},r\varphi)
  \right)
  \partial_{r}
  \psi^{n}(\hat{x}_{j},r\varphi)
  \mathrm{d}r
  \mathrm{d}\varphi
  \mathrm{d}^{2(n-1)}\hat{x}_{j}
  \\
  &\phantom{=\ }
  +
  \lim_{r \downarrow 0}
  \int_{\mathbb{R}^{2(n-1)}}
  \int_{S^{1}}
  \overline{\phi^{n}}(\hat{x}_{j},r\varphi)
  \psi^{n}(\hat{x}_{j},r\varphi)
  \mathrm{d}\varphi
  \mathrm{d}^{2(n-1)}\hat{x}_{j}
  \\
  &\phantom{=\ }
  +
  \int_{\mathbb{R}^{2(n-1)}}
  \int_{S^{1}}
  \int_{0}^{\infty}
  \left(
    \partial_{r}
    \overline{\phi^{n}}(\hat{x}_{j},r\varphi)
  \right)
  \psi^{n}(\hat{x}_{j},r\varphi)
  \mathrm{d}r
  \mathrm{d}\varphi
  \mathrm{d}^{2(n-1)}\hat{x}_{j}
  \\
  &=
  \lim_{r \downarrow 0}
  \int_{\mathbb{R}^{2(n-1)}}
  \int_{S^{1}}
  r
  \overline{\phi^{n}}(\hat{x}_{j},r\varphi)
  \partial_{r}
  \psi^{n}(\hat{x}_{j},r\varphi)
  \mathrm{d}\varphi
  \mathrm{d}^{2(n-1)}\hat{x}_{j}
  \\
  &\phantom{=\ }
  -
  \lim_{r \downarrow 0}
  \int_{\mathbb{R}^{2(n-1)}}
  \int_{S^{1}}
  \left(
    \partial_{r}
    r
    \overline{\phi^{n}}(\hat{x}_{j},r\varphi)
  \right)
  \psi^{n}(\hat{x}_{j},r\varphi)
  \mathrm{d}\varphi
  \mathrm{d}^{2(n-1)}\hat{x}_{j}
  \\
  &\phantom{=\ }
  -
  \int_{\mathbb{R}^{2(n-1)}}
  \int_{S^{1}}
  \int_{0}^{\infty}
  \left(
    \partial_{r}^{2}
    r
    \overline{\phi^{n}}(\hat{x}_{j},r\varphi)
  \right)
  \psi^{n}(\hat{x}_{j},r\varphi)
  \mathrm{d}r
  \mathrm{d}\varphi
  \mathrm{d}^{2(n-1)}\hat{x}_{j}
  \\
  &\phantom{=\ }
  +
  \lim_{r \downarrow 0}
  \int_{\mathbb{R}^{2(n-1)}}
  \int_{S^{1}}
  \overline{\phi^{n}}(\hat{x}_{j},r\varphi)
  \psi^{n}(\hat{x}_{j},r\varphi)
  \mathrm{d}\varphi
  \mathrm{d}^{2(n-1)}\hat{x}_{j}
  \\
  &\phantom{=\ }
  +
  \int_{\mathbb{R}^{2(n-1)}}
  \int_{S^{1}}
  \int_{0}^{\infty}
  \left(
    \partial_{r}
    \overline{\phi^{n}}(\hat{x}_{j},r\varphi)
  \right)
  \psi^{n}(\hat{x}_{j},r\varphi)
  \mathrm{d}r
  \mathrm{d}\varphi
  \mathrm{d}^{2(n-1)}\hat{x}_{j}
  \\
  &=
  \lim_{r \downarrow 0}
  \int_{\mathbb{R}^{2(n-1)}}
  \int_{S^{1}}
  r
  \overline{\phi^{n}}(\hat{x}_{j},r\varphi)
  \partial_{r}
  \psi^{n}(\hat{x}_{j},r\varphi)
  \mathrm{d}\varphi
  \mathrm{d}^{2(n-1)}\hat{x}_{j}
  \\
  &\phantom{=\ }
  -
  \lim_{r \downarrow 0}
  \int_{\mathbb{R}^{2(n-1)}}
  \int_{S^{1}}
  r
  \left(
    \partial_{r}
    \overline{\phi^{n}}(\hat{x}_{j},r\varphi)
  \right)
  \psi^{n}(\hat{x}_{j},r\varphi)
  \mathrm{d}\varphi
  \mathrm{d}^{2(n-1)}\hat{x}_{j}
  \\
  &\phantom{=\ }
  -
  \int_{\mathbb{R}^{2(n-1)}}
  \int_{S^{1}}
  \int_{0}^{\infty}
  \left(
    2
    \partial_{r}
    \overline{\phi^{n}}(\hat{x}_{j},r\varphi)
    +
    r
    \partial_{r}^{2}
    \overline{\phi^{n}}(\hat{x}_{j},r\varphi)
  \right)
  \psi^{n}(\hat{x}_{j},r\varphi)
  \mathrm{d}r
  \mathrm{d}\varphi
  \mathrm{d}^{2(n-1)}\hat{x}_{j}
  \\
  &\phantom{=\ }
  +
  \int_{\mathbb{R}^{2(n-1)}}
  \int_{S^{1}}
  \int_{0}^{\infty}
  \left(
    \partial_{r}
    \overline{\phi^{n}}(\hat{x}_{j},r\varphi)
  \right)
  \psi^{n}(\hat{x}_{j},r\varphi)
  \mathrm{d}r
  \mathrm{d}\varphi
  \mathrm{d}^{2(n-1)}\hat{x}_{j}
  \\
  &=
  \lim_{r \downarrow 0}
  \int_{\mathbb{R}^{2(n-1)}}
  \int_{S^{1}}
  r
  \overline{\phi^{n}}(\hat{x}_{j},r\varphi)
  \partial_{r}
  \psi^{n}(\hat{x}_{j},r\varphi)
  \mathrm{d}\varphi
  \mathrm{d}^{2(n-1)}\hat{x}_{j}
  \\
  &\phantom{=\ }
  -
  \lim_{r \downarrow 0}
  \int_{\mathbb{R}^{2(n-1)}}
  \int_{S^{1}}
  r
  \left(
    \partial_{r}
    \overline{\phi^{n}}(\hat{x}_{j},r\varphi)
  \right)
  \psi^{n}(\hat{x}_{j},r\varphi)
  \mathrm{d}\varphi
  \mathrm{d}^{2(n-1)}\hat{x}_{j}
  \\
  &\phantom{=\ }
  -
  \int_{\mathbb{R}^{2(n-1)}}
  \int_{S^{1}}
  \int_{0}^{\infty}
  r
  \partial_{r}^{2}
  \overline{\phi^{n}}(\hat{x}_{j},r\varphi)
  \psi^{n}(\hat{x}_{j},r\varphi)
  \mathrm{d}r
  \mathrm{d}\varphi
  \mathrm{d}^{2(n-1)}\hat{x}_{j}
  \\
  &\phantom{=\ }
  -
  \int_{\mathbb{R}^{2(n-1)}}
  \int_{S^{1}}
  \int_{0}^{\infty}
  r
  \left(
    \frac{1}{r}
    \partial_{r}
    \overline{\phi^{n}}(\hat{x}_{j},r\varphi)
  \right)
  \psi^{n}(\hat{x}_{j},r\varphi)
  \mathrm{d}r
  \mathrm{d}\varphi
  \mathrm{d}^{2(n-1)}\hat{x}_{j}
  \\
  &=
  \lim_{r \downarrow 0}
  \int_{\mathbb{R}^{2(n-1)}}
  \int_{S^{1}}
  r
  \frac{\ln(r)}{\ln(r)}
  \overline{\phi^{n}}(\hat{x}_{j},r\varphi)
  \partial_{r}
  \frac{\ln(r)}{\ln(r)}
  \psi^{n}(\hat{x}_{j},r\varphi)
  \mathrm{d}\varphi
  \mathrm{d}^{2(n-1)}\hat{x}_{j}
  \\
  &\phantom{=\ }
  -
  \lim_{r \downarrow 0}
  \int_{\mathbb{R}^{2(n-1)}}
  \int_{S^{1}}
  r
  \left(
    \partial_{r}
    \frac{\ln(r)}{\ln(r)}
    \overline{\phi^{n}}(\hat{x}_{j},r\varphi)
  \right)
  \frac{\ln(r)}{\ln(r)}
  \psi^{n}(\hat{x}_{j},r\varphi)
  \mathrm{d}\varphi
  \mathrm{d}^{2(n-1)}\hat{x}_{j}
  \\
  &\phantom{=\ }
  -
  \int_{\mathbb{R}^{2n}}
  \left(
    (\Delta_{j} - L_{j})
    \overline{\phi^{n}}
    \left(
      x_{1}
      ,
      \dots
      ,
      x_{n}
    \right)
  \right)
  \psi^{n}(x_{1},\dots,x_{n})
  \mathrm{d}^{2n}x
  \\
  &=
  \lim_{r \downarrow 0}
  \int_{\mathbb{R}^{2(n-1)}}
  \int_{S^{1}}
  \frac{1}{\ln(r)}
  \overline{\phi^{n}}(\hat{x}_{j},r\varphi)
  \psi^{n}(\hat{x}_{j},r\varphi)
  \mathrm{d}\varphi
  \mathrm{d}^{2(n-1)}\hat{x}_{j}
  \\
  &\phantom{=\ }
  +
  \lim_{r \downarrow 0}
  \int_{\mathbb{R}^{2(n-1)}}
  \int_{S^{1}}
  \frac{1}{\ln(r)}
  \overline{\phi^{n}}(\hat{x}_{j},r\varphi)
  r
  \ln^{2}(r)
  \partial_{r}
  \frac{1}{\ln(r)}
  \psi^{n}(\hat{x}_{j},r\varphi)
  \mathrm{d}\varphi
  \mathrm{d}^{2(n-1)}\hat{x}_{j}
  \\
  &\phantom{=\ }
  -
  \lim_{r \downarrow 0}
  \int_{\mathbb{R}^{2(n-1)}}
  \int_{S^{1}}
  \frac{1}{\ln(r)}
  \overline{\phi^{n}}(\hat{x}_{j},r\varphi)
  \psi^{n}(\hat{x}_{j},r\varphi)
  \mathrm{d}\varphi
  \mathrm{d}^{2(n-1)}\hat{x}_{j}
  \\
  &\phantom{=\ }
  -
  \lim_{r \downarrow 0}
  \int_{\mathbb{R}^{2(n-1)}}
  \int_{S^{1}}
  r
  \ln^{2}(r)
  \left(
    \partial_{r}
    \frac{1}{\ln(r)}
    \overline{\phi^{n}}(\hat{x}_{j},r\varphi)
  \right)
  \frac{1}{\ln(r)}
  \psi^{n}(\hat{x}_{j},r\varphi)
  \mathrm{d}\varphi
  \mathrm{d}^{2(n-1)}\hat{x}_{j}
  \\
  &\phantom{=\ }
  -
  \int_{\mathbb{R}^{2n}}
  \left(
    (\Delta_{j} - L_{j})
    \overline{\phi^{n}}
    \left(
      x_{1}
      ,
      \dots
      ,
      x_{n}
    \right)
  \right)
  \psi^{n}(x_{1},\dots,x_{n})
  \mathrm{d}^{2n}x
  \\
  &=
  \int_{\mathbb{R}^{2(n-1)}}
  \overline{\phi^{n-1}}(\hat{x}_{j})
  \left(
    \frac{\overline{c}}{2\pi\sqrt{n}}
    \lim_{r \downarrow 0}
    r
    \ln^{2}(r)
    \int_{S^{1}}
    \partial_{r}
    \frac{1}{\ln(r)}
    \psi^{n}(\hat{x}_{j},r\varphi)
    \mathrm{d}\varphi
  \right)
  \mathrm{d}^{2(n-1)}\hat{x}_{j}
  \\
  &\phantom{=\ }
  -
  \int_{\mathbb{R}^{2(n-1)}}
  \left(
    \frac{c}{2\pi\sqrt{n}}
    \lim_{r \downarrow 0}
    r
    \ln^{2}(r)
    \int_{S^{1}}
    \partial_{r}
    \frac{1}{\ln(r)}
    \overline{\phi^{n}}(\hat{x}_{j},r\varphi)
    \mathrm{d}\varphi
  \right)
  \psi^{n-1}(\hat{x}_{j})
  \mathrm{d}^{2(n-1)}\hat{x}_{j}
  \\
  &\phantom{=\ }
  -
  \int_{\mathbb{R}^{2n}}
  \left(
    (\Delta_{j} - L_{j})
    \overline{\phi^{n}}
    \left(
      x_{1}
      ,
      \dots
      ,
      x_{n}
    \right)
  \right)
  \psi^{n}
  \left(
    x_{1}
    ,
    \dots
    ,
    x_{n}
  \right)
  \mathrm{d}^{2n}x
  \,.
\end{align*}
In the last step we used the IBC and the symmetry of $\phi,\psi$ in their arguments in each sector. The latter property we use now to conclude that the first two terms after the last sign of equality do not depend on $j$, so that we can write $\hat{x}_{n}$ instead of $\hat{x}_{j}$ everywhere in these terms. From the above calculation it is hence immediate that
\begin{align*}
  &
  -
  \sum_{j = 1}^{n}
  \int_{\mathbb{R}^{2n}}
  \overline{\phi^{n}}
  \left(
    x_{1}
    ,
    \dots
    ,
    x_{n}
  \right)
  (\Delta_{j} - L_{j})
  \psi^{n}
  \left(
    x_{1}
    ,
    \dots
    ,
    x_{n}
  \right)
  \mathrm{d}^{2n}x
  \\
  &=
  \int_{\mathbb{R}^{2(n-1)}}
  \overline{\phi^{n-1}}(\hat{x}_{n})
  \left(
    \frac{\overline{c}\sqrt{n}}{2\pi}
    \lim_{r \downarrow 0}
    r
    \ln^{2}(r)
    \int_{S^{1}}
    \partial_{r}
    \frac{1}{\ln(r)}
    \psi^{n}(\hat{x}_{n},r\varphi)
    \mathrm{d}\varphi
  \right)
  \mathrm{d}^{2(n-1)}\hat{x}_{n}
  \\
  &\phantom{=\ }
  -
  \int_{\mathbb{R}^{2(n-1)}}
  \left(
    \frac{c\sqrt{n}}{2\pi}
    \lim_{r \downarrow 0}
    r
    \ln^{2}(r)
    \int_{S^{1}}
    \partial_{r}
    \frac{1}{\ln(r)}
    \overline{\phi^{n}}(\hat{x}_{n},r\varphi)
    \mathrm{d}\varphi
  \right)
  \psi^{n-1}(\hat{x}_{n})
  \mathrm{d}^{2(n-1)}\hat{x}_{n}
  \\
  &\phantom{=\ }
  -
  \sum_{j = 1}^{n}
  \int_{\mathbb{R}^{2n}}
  \left(
    (\Delta_{j} - L_{j})
    \overline{\phi^{n}}
    \left(
      x_{1}
      ,
      \dots
      ,
      x_{n}
    \right)
  \right)
  \psi^{n}(x_{1},\dots,x_{n})
  \mathrm{d}^{2n}x
  \\
  &=
  -
  \int_{\mathbb{R}^{2(n-1)}}
  \overline{\phi^{n-1}}(\hat{x}_{n})
  \overline{c}
  \left(
    \mathcal{A}
    \psi
  \right)^{n - 1}
  (\hat{x}_{n})
  \mathrm{d}^{2(n-1)}\hat{x}_{n}
  +
  \int_{\mathbb{R}^{2(n-1)}}
  \overline{
  \overline{c}
  \left(
    \mathcal{A}
    \phi
  \right)^{n - 1}
  (\hat{x}_{n})
  }
  \psi^{n-1}(\hat{x}_{n})
  \mathrm{d}^{2(n-1)}\hat{x}_{n}
  \\
  &\phantom{=\ }
  -
  \sum_{j = 1}^{n}
  \int_{\mathbb{R}^{2n}}
  \left(
    (\Delta_{j} - L_{j})
    \overline{\phi^{n}}
    \left(
      x_{1}
      ,
      \dots
      ,
      x_{n}
    \right)
  \right)
  \psi^{n}
  \left(
    x_{1}
    ,
    \dots
    ,
    x_{n}
  \right)
  \mathrm{d}^{2n}x
  \\
  &=
  -
  \left(
      \phi^{n-1}
    \big\vert
      \overline{c}
      \left(
        \mathcal{A}
        \psi
      \right)^{n - 1}
  \right)_{n - 1}
  +
  \left(
      \overline{c}
      \left(
        \mathcal{A}
        \phi
      \right)^{n - 1}
    \big\vert
      \psi^{n - 1}
  \right)_{n - 1}
  \\
  &\phantom{=\ }
  -
  \sum_{j = 1}^{n}
  \int_{\mathbb{R}^{2n}}
  \left(
    (\Delta_{j} - L_{j})
    \overline{\phi^{n}}
    \left(
      x_{1}
      ,
      \dots
      ,
      x_{n}
    \right)
  \right)
  \psi^{n}
  \left(
    x_{1}
    ,
    \dots
    ,
    x_{n}
  \right)
  \mathrm{d}^{2n}x
  \,.
\end{align*}
With the symmetry of $L_{j}$ thus follows
\begin{align*}
  \left(
      \phi^{n}
    \vert
      (-\Delta^{\mathcal{F}}\psi)^{n}
  \right)_{n}
  &=
  -
  \left(
      \phi^{n-1}
    \big\vert
      \overline{c}
      \left(
        \mathcal{A}
        \psi
      \right)^{n - 1}
  \right)_{n - 1}
  +
  \left(
      \overline{c}
      \left(
        \mathcal{A}
        \phi
      \right)^{n - 1}
    \big\vert
      \psi^{n - 1}
  \right)_{n - 1}
  +
  \left(
      (-\Delta^{\mathcal{F}}\phi)^{n}
    \vert
      \psi^{n}
  \right)_{n}
  \,.
\end{align*}
This, in a way, is a generalized integration-by-parts formula. The asymmetry of $H$ in each sector can hence be described in terms of the operator $\mathcal{A}$ and will cancel out by summing over all $n \in \mathbb{N}$. We calculate
\begin{align*}
  \left(
      \phi
    \vert
      H
      \psi
  \right)_{\mathcal{F}}
  &=
  \left(
      \phi
    \vert
      \mathrm{d}\Gamma(h)
      \psi
  \right)_{\mathcal{F}}
  +
  \left(
      \phi
    \vert
      \overline{c}
      \mathcal{A}
      \psi
  \right)_{\mathcal{F}}
  \\
  &=
  \sum_{n \in \mathbb{N}^{\times}}
  \left(
      \phi^{n}
    \vert
      \left(
        -
        \Delta^{\mathcal{F}}
        \psi
      \right)^{n}
  \right)_{n}
  +
  \left(
      \phi
    \vert
      \mathrm{d}\Gamma(E_{0})
      \psi
  \right)_{\mathcal{F}}
  +
  \left(
      \phi
    \vert
      \overline{c}
      \mathcal{A}
      \psi
  \right)_{\mathcal{F}}
  \\
  &=
  \sum_{n \in \mathbb{N}^{\times}}
  \left(
    \left(
        \left(
          -
          \Delta^{\mathcal{F}}
          \phi
        \right)^{n}
      \vert
        \psi^{n}
    \right)_{n}
    -
    \left(
          \phi^{n-1}
      \big\vert
        \overline{c}
        \left(
          \mathcal{A}
          \psi
        \right)^{n - 1}
    \right)_{n - 1}
    +
    \left(
        \overline{c}
        \left(
          \mathcal{A}
          \phi
        \right)^{n - 1}
      \big\vert
        \psi^{n - 1}
    \right)_{n - 1}
  \right)
  \\
  &\phantom{=\ }
  +
  \left(
      \mathrm{d}\Gamma(E_{0})
      \phi
    \vert
      \psi
  \right)_{\mathcal{F}}
  +
  \left(
      \phi
    \vert
      \overline{c}
      \mathcal{A}
      \psi
  \right)_{\mathcal{F}}
  \\
  &=
  \left(
      -
      \Delta^{\mathcal{F}}
      \phi
    \vert
      \psi
  \right)_{\mathcal{F}}
  -
  \left(
      \phi
    \vert
      \overline{c}
      \mathcal{A}
      \psi
  \right)_{\mathcal{F}}
  +
  \left(
      \overline{c}
      \mathcal{A}
      \phi
    \vert
      \psi
  \right)_{\mathcal{F}}
  +
  \left(
      \mathrm{d}\Gamma(E_{0})
      \phi
    \vert
      \psi
  \right)_{\mathcal{F}}
  +
  \left(
      \phi
    \vert
      \overline{c}
      \mathcal{A}\psi
  \right)_{\mathcal{F}}
  \\
  &=
  \left(
      H
      \phi
    \vert
      \psi
  \right)_{\mathcal{F}}
  \,.
\end{align*}
