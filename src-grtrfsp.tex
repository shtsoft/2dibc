As we only have the zero- and the one-particle sector here, we write $\psi = (\psi^{0},\psi^{1})$ and the model reads
\begin{align*}
  (H\psi)^{0}
  &=
  -
  \frac{\overline{c}}{2\pi}
  \lim_{r \downarrow 0}
  r\ln^{2}(r)
  \partial_{r}
  \frac{1}{\ln(r)}
  \int_{S^{1}}
  \psi^{1}
  \left(
    r\varphi
  \right)
  \mathrm{d}\varphi
  ,\qquad
  (H\psi)^{1}
  =
  -
  \Delta
  \psi^{1}
\end{align*}
with IBC
\begin{align*}
  c
  \psi^{0}
  =
  2\pi
  \lim_{r \downarrow 0}
  \frac{1}{\ln(r)}
  \psi^{1}
  \left(
    r\varphi
  \right)
  \,.
\end{align*}
To calculate the ground state for the eigenvalue equation $H\psi = E\psi$ we assume $E = E_{g} < 0$, i.e. that the ground state has negative energy $E_{g}$. $\psi^{1}$ has to solve $(\Delta + E_{g})\psi^{1} = 0$ away from the origin and we make the ansatz $\psi_{g}^{1}(x) = Af_{\kappa}(x) := Af_{\kappa,0}(x) = AK_{0}(\kappa\Vert x \Vert)$ with $\kappa > 0$ and $A \in \mathbb{C}^{\times}$. This function solves the eigenvalue equation in the one particle sector with $E_{g} = -\kappa^{2}$ as we saw above by using polar coordinates. The IBC yields
\begin{align*}
  \psi_{g}^{0}
  &=
  -
  A
  \frac{2\pi}{c}
  \,,
\end{align*}
and the eigenvalue equation in the zero-particle sector $(H\psi)^{0} = E_{g}\psi^{0}$ demands
\begin{align*}
  A
  \overline{c}
  \left(
    -
    \gamma
    +
    \ln(2)
    -
    \ln(\kappa)
  \right)
  =
  A\kappa^{2}
  \frac{2\pi}{c}
  \qquad
  &\Leftrightarrow
  \qquad
  \ln(\kappa)
  =
  -
  \kappa^{2}
  \frac{2\pi}{\vert c \vert^{2}}
  -
  \gamma
  +
  \ln(2)
  \,.
\end{align*}
Now $\ln(\kappa)$ is strictly increasing and $-\kappa^{2}$ is strictly decreasing for $\kappa > 0$, with $\ln(\kappa) \to \infty$ and $-\kappa^{2} \to -\infty$ as $\kappa \to \infty$. Additionally the right side of the above equation is greater than the left side for $\kappa$ small enough. Thus, there is a unique solution in $\mathbb{R}^{+}$ for this equation. In fact, the solution can be given using the so called product Lambert $W$ function (see e.g. \cite{1af276f4}) or, to be more accurate, its principal branch $W_{0}$. The result is
\begin{align*}
  \kappa
  &=
  \left(
    \frac{\vert c \vert^{2}}{4\pi}
    W_{0}
    \left(
      \frac{4\pi}{\vert c \vert^{2}}
      \exp(2(\ln(2) - \gamma))
    \right)
  \right)^{\frac{1}{2}}
  \,,
\end{align*}
which can easily be checked using the identity
\begin{align*}
  \ln(W_{0}(z))
  &=
  \ln(z)
  -
  W_{0}(z)
  ,\qquad
  z > 0
  \,.
\end{align*}
Indeed, we find
\begin{align*}
  \ln(\kappa)
  &=
  \frac{1}{2}
  \left(
    \ln
    \left(
      \frac{\vert c \vert^{2}}{4\pi}
    \right)
    +
    \ln
    \left(
      W_{0}
      \left(
        \frac{4\pi}{\vert c \vert^{2}}
        \exp(2(\ln(2) - \gamma))
      \right)
    \right)
  \right)
  \\
  &=
  \frac{1}{2}
  \left(
    -
    \ln
    \left(
      \frac{4\pi}{\vert c \vert^{2}}
    \right)
    +
    \ln
    \left(
      \frac{4\pi}{\vert c \vert^{2}}
      \exp(2(\ln(2) - \gamma))
    \right)
    -
    W_{0}
    \left(
      \frac{4\pi}{\vert c \vert^{2}}
      \exp(2(\ln(2) - \gamma))
    \right)
  \right)
  \\
  &=
  \ln(2)
  -
  \gamma
  -
  \frac{1}{2}
  \frac{4\pi}{\vert c \vert^{2}}
  \frac{\vert c \vert^{2}}{4\pi}
  W_{0}
  \left(
    \frac{4\pi}{\vert c \vert^{2}}
    \exp(2(\ln(2) - \gamma))
  \right)
  \\
  &=
  \ln(2)
  -
  \gamma
  -
  \frac{2\pi}{\vert c \vert^{2}}
  \kappa^{2}
  \,.
\end{align*}
From the normalization condition we can finally deduce the constant $A$ using polar coordinates, where the angular integration can be done trivially. We find
\begin{align*}
  1
  =
  \left(
    \psi_{g}
    \vert
    \psi_{g}
  \right)_{\mathcal{F}_{1}}
  &=
  \overline{\psi_{g}^{0}}
  \psi_{g}^{0}
  +
  \vert A \vert^{2}
  \int_{\mathbb{R}^{2}}
  \overline{f_{\kappa}(x)}
  f_{\kappa}(x)
  \mathrm{d}x
  \\
  &=
  \vert A \vert^{2}
  \frac{4\pi^{2}}{\vert c \vert^{2}}
  +
  2\pi
  \vert A \vert^{2}
  \int_{0}^{\infty}
  r
  K_{0}^{2}(\kappa r)
  \mathrm{d}r
  \\
  &=
  \vert A \vert^{2}
  \frac{4\pi^{2}}{\vert c \vert^{2}}
  +
  \frac{2\pi}{\kappa^{2}}
  \vert A \vert^{2}
  I
  \\
  \Rightarrow
  \qquad
  \vert A \vert^{2}
  &=
  \left(
    \frac{4\pi^{2}}{\vert c \vert^{2}}
    +
    \frac{2\pi}{\kappa^{2}}
    I
  \right)^{-1}
  \,,
\end{align*}
where we substituted $z = \kappa r$ and the resulting integral is
\begin{align*}
  I
  &:=
  \int_{0}^{\infty}
  z
  K_{0}^{2}(z)
  \mathrm{d}z
  =
  \frac{1}{2}
  \,.
\end{align*}
To see this, we use the recurrence relations (see e.g. \cite{35763ba8})
\begin{align*}
  \mathcal{Z}_{\alpha}^{\prime}(z)
  &=
  \mathcal{Z}_{\alpha - 1}(z)
  -
  \frac{\alpha}{z}
  \mathcal{Z}_{\alpha}(z)
  \,,
  \\
  \mathcal{Z}_{\alpha}^{\prime}(z)
  &=
  \mathcal{Z}_{\alpha + 1}(z)
  +
  \frac{\alpha}{z}
  \mathcal{Z}_{\alpha}(z)
  \,,
\end{align*}
where $\mathcal{Z}_{\alpha} := \exp(\mathrm{i}\pi\alpha)K_{\alpha}$. With these relations we can calculate
\begin{align*}
  \left(
    \frac{z^{2}}{2}
    \left(
      K_{0}^{2}(z)
      -
      K_{1}^{2}(z)
    \right)
  \right)^{\prime}
  &=
  z
  \left(
    K_{0}^{2}(z)
    -
    K_{1}^{2}(z)
  \right)
  +
  \frac{z^{2}}{2}
  \left(
    K_{0}^{2}(z)
    -
    K_{1}^{2}(z)
  \right)^{\prime}
  \\
  &=
  z
  \left(
    K_{0}^{2}(z)
    -
    K_{1}^{2}(z)
  \right)
  +
  z^{2}
  \left(
    K_{0}(z)
    K_{0}^{\prime}(z)
    -
    K_{1}(z)
    K_{1}^{\prime}(z)
  \right)
  \\
  &=
  z
  \left(
    K_{0}^{2}(z)
    -
    K_{1}^{2}(z)
  \right)
  +
  z^{2}
  \left(
    -
    K_{0}(z)
    K_{1}(z)
    +
    K_{1}(z)
    K_{0}(z)
    +
    \frac{1}{z}
    K_{1}^{2}(z)
  \right)
  \\
  &=
  z
  K_{0}^{2}(z)
  \,.
\end{align*}
Now (using the relations from \cite{ca427945},\cite{69346099}),
\begin{align*}
  \lim_{z \to \infty}
  z
  K_{\alpha}(z)
  &=
  0
  \,,
  \qquad
  \lim_{z \to 0}
  z
  K_{0}(z)
  =
  0
  \qquad
  \text{and}
  \qquad
  \lim_{z \to 0}
  z
  K_{1}(z)
  =
  1
  \,,
\end{align*}
and thus
\begin{align*}
  I
  &=
  \lim_{z \to \infty}
  \frac{z^{2}}{2}
  \left(
    K_{0}^{2}(z)
    -
    K_{1}^{2}(z)
  \right)
  -
  \lim_{z \to 0}
  \frac{z^{2}}{2}
  \left(
    K_{0}^{2}(z)
    -
    K_{1}^{2}(z)
  \right)
  =
  \frac{1}{2}
  \,.
\end{align*}
\\
For vanishing coupling, $\vert c \vert \to 0$, the free vacuum state $\psi_{\textrm{vac}} = (1,0) \in \mathcal{F}_{1}$ with energy $E_{\textrm{vac}} = 0$ is recovered, up to an undetermined phase. This can be seen by using that
\begin{align*}
  \lim_{z \to \infty}
  W_{0}(z)
  &=
  \lim_{z \to \infty}
  \ln(z)
  \,.
\end{align*}
We find
\begin{align*}
  \vert \psi_{g}^{0} \vert^{2}
  &=
  \frac{4\pi^{2}}{\vert c \vert^{2}}
  \left(
    \frac{4\pi^{2}}{\vert c \vert^{2}}
    +
    \frac{\pi}{\kappa^{2}}
  \right)^{-1}
  =
  \left(
    1
    +
    \frac{\vert c \vert^{2}}{4\pi}
    \frac{1}{\kappa^{2}}
  \right)^{-1}
  =
  \left(
    1
    +
    \left(
      W_{0}
      \left(
        \frac{4\pi}{\vert c \vert^{2}}
        \exp(2(\ln(2) - \gamma))
      \right)
    \right)^{-1}
  \right)^{-1}
\end{align*}
and
\begin{align*}
  \Vert \psi_{g}^{1} \Vert_{L^{2}}^{2}
  &=
  \frac{\pi}{\kappa^{2}}
  \left(
    \frac{4\pi^{2}}{\vert c \vert^{2}}
    +
    \frac{\pi}{\kappa^{2}}
  \right)^{-1}
  =
  \left(
    \frac{4\pi}{\vert c \vert^{2}}
    \kappa^{2}
    +
    1
  \right)^{-1}
  =
  \left(
    W_{0}
    \left(
      \frac{4\pi}{\vert c \vert^{2}}
      \exp(2(\ln(2) - \gamma))
    \right)
    +
    1
  \right)^{-1}
  \,.
\end{align*}
In the limit $\vert c \vert \to 0$ we have $\alpha := \frac{4\pi}{\vert c \vert^{2}}\exp(2(\ln(2) - \gamma)) \to \infty$ and hence
\begin{align*}
  \lim_{\vert c \vert \to 0}
  \vert \psi_{g}^{0} \vert^{2}
  =
  1
  &,\qquad
  \lim_{\vert c \vert \to 0}
  \Vert \psi_{g}^{1} \Vert_{L^{2}}^{2}
  =
  0
  \,.
\end{align*}
Finally,
\begin{align*}
  \lim_{\vert c \vert \to 0}
  E_{g}
  &=
  -
  \lim_{\vert c \vert \to 0}
  \frac{\vert c \vert^{2}}{4\pi}
  W_{0}
  \left(
    \frac{4\pi}{\vert c \vert^{2}}
    \exp(2(\ln(2) - \gamma))
  \right)
  \\
  &=
  -
  \exp(2(\ln(2) - \gamma))
  \lim_{\alpha \to \infty}
  \frac{\ln(\alpha)}{\alpha}
  \\
  &=
  0
  \,.
\end{align*}
This result is different from the one-dimensional case (cf. \cite{3d622ca2}), where the limit of the ground state is proportional to the vacuum state, but the normalization is not preserved. This one-dimensional result stems from an infrared divergence on full Fock space.
