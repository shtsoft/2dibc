\nocite{d3d4ed0b}
%%%
We still want to give mathematical meaning to a Hamilton operator that, in a way, captures the physical properties of $H_{\delta_{y}}$ without the need for renormalization. We do this by means of an interior-boundary condition. Following the motivation from \cite{3d622ca2} we write out the formal eigenvalue equation $H_{\delta_{y}}\psi = E\psi$ in the $n$-particle sector ($n \geq 1$)
\begin{align}
\label{eveq}
  &
  -
  \sum_{j = 1}^{n}
  \Delta_{j}\psi^{n}
  \left(
    x_{1}
    ,
    \dots
    ,
    x_{n}
  \right)
  +
  \overline{c}
  \sqrt{n + 1}
  \psi^{n + 1}
  \left(
    x_{1}
    ,
    \dots
    ,
    x_{n}
    ,
    y
  \right)
  +
  \frac{c}{\sqrt{n}}
  \sum_{j = 1}^{n}
  \delta_{y}(x_{j})
  \psi^{n - 1}
  \left(
    x_{1}
    ,
    \dots
    ,
    \hat{x}_{j}
    ,
    \dots,x_{n}
  \right)
  \\
  &=
  E
  \psi^{n}
  \left(
    x_{1}
    ,
    \dots
    ,
    x_{n}
  \right)
  \,.
\nonumber
\end{align}
Here we formally wrote $\delta_{y}(x_{j})$ assuming there is a function (which is actually not true) $\delta_{y} \colon \mathbb{R}^{2} \to [-\infty,\infty]$ vanishing on $\mathbb{R}^{2} \setminus \lbrace y \rbrace$ with the property
\begin{align*}
  \int_{\mathbb{R}^{2}}
  \delta_{y}(x)
  \mathrm{d}x
  &=
  1
  \,.
\end{align*}
Now we integrate \eqref{eveq} in\footnote{Remember that the $\psi^{n}$ are symmetric so that it does not matter which coordinate is integrated.} $x_{n} \in \mathbb{R}^{2}$ over the ball
\begin{align*}
  \bar{\mathbb{B}}(y,r)
  &:=
  \left\lbrace
    x
    \in
    \mathbb{R}^{2}
    \colon
    \Vert
      x
      -
      y
    \Vert
    \leq
    r
  \right\rbrace
  \,,
\end{align*}
where $y \in \mathbb{R}^{2}$ and $r \in \mathbb{R}^{+}$. We assume that the $\psi^{n}$ for $n \in \mathbb{N}^{\times}$ are sufficiently regular in $y$, such that
\begin{align*}
  \int_{\bar{\mathbb{B}}(y,r)}
  \psi^{n}
  \left(
    x_{1}
    ,
    \dots
    ,
    x_{n}
  \right)
  \mathrm{d}x_{n}
  \to
  0
  \qquad
  \text{as}
  \qquad
  r
  \downarrow
  0
  \,,
\end{align*}
but that this is not necessarily true for the derivatives. Taking the limit $r \downarrow 0$ after the integration then yields for all $x_{1},\dots,x_{n-1} \in \mathbb{R}^{2}$,
\begin{align}
\label{ibcbase}
  -
  \lim_{r \downarrow 0}
  \int_{\bar{\mathbb{B}}(y,r)}
  \Delta_{n}\psi^{n}
  \left(
    x_{1}
    ,
    \dots
    ,
    x_{n}
  \right)
  \mathrm{d}x_{n}
  +
  \frac{c}{\sqrt{n}}
  \psi^{n-1}
  \left(
    x_{1}
    ,
    \dots
    ,
    x_{n-1}
  \right)
  &=
  0
  \,.
\end{align}
This equation basically describes what the creation operator does and is the foundation for our IBC. We use Gau{\ss}' theorem to rewrite the first term as
\begin{align}
\label{polint}
  -
  \int_{\bar{\mathbb{B}}(y,r)}
  \Delta_{n}\psi^{n}
  \left(
    x_{1}
    ,
    \dots
    ,
    x_{n}
  \right)
  \mathrm{d}x_{n}
  &=
  -
  \int_{S^{1}}
  r
  \partial_{r}
  \psi^{n}
  \left(
    x_{1}
    ,
    \dots
    ,
    x_{n-1}
    ,
    y
    +
    r
    \varphi
  \right)
  \mathrm{d}\varphi
  \,,
\end{align}
where we have introduced polar coordinates $(r,\varphi) \in [0,\infty) \times S^{1}$ with $S^{1} \subset \mathbb{R}^{2}$ being the $1$-sphere such that for $x \in \mathbb{R}^{2}$ we have $r = \Vert x \Vert$ and $\varphi = x/\Vert x \Vert$. Now the integral in \eqref{polint} vanishes for sufficiently regular $\psi^{n}$ in the limit $r \downarrow 0$, but we want a non-vanishing and finite result for our IBC to make sense for non-vanishing and finite $\psi^{n-1}$. Thus, we are looking for a function to add to the domain of the one-particle sector that diverges like $\ln(r)$ as $r \downarrow 0$. Still, away from $y$, this function should suffice the Schr\"odinger eigenvalue equation.
\\
To find such a function we write the Laplace operator in two dimensions in polar coordinates, where we assume that the function $f \colon \mathbb{R}^{2} \to \mathbb{C}$, the Laplace operator acts on, does not depend on the {\glqq}direction{\grqq}, i.e. on $\varphi$, but only on the {\glqq}distance from the source{\grqq}\footnote{This can be reasoned, e.g., from a physical point of view, as we have a rotational symmetry around the source.}, i.e. on $r$. Therefore we omit the angular part of the Laplace operator and then have
\begin{align*}
  \Delta f
  &=
  \partial_{r}^{2}
  f
  +
  \frac{1}{r}
  \partial_{r}
  f
  \,.
\end{align*}
The eigenvalue equation hence reads
\begin{align*}
  \partial_{r}^{2}
  f
  +
  \frac{1}{r}
  \partial_{r}
  f
  -
  \kappa^{2}
  f
  &=
  0
  \,,
\end{align*}
where we have written the energy as $E = -\kappa^{2}$ for $\kappa > 0$, which we assume here to be the energy for the ground state. This equation resembles the modified Bessel differential equation\footnote{For more on Bessel's differential equations, the Bessel functions and the relations used here see e.g. \cite{29f54c49} and \cite{ca427945},\cite{69346099},\cite{35763ba8}.} for $z > 0$ and $g \colon (0,\infty) \to \mathbb{R}$,
\begin{align*}
  z^{2}
  \partial_{z}^{2}
  g(\kappa z)
  +
  z
  \partial_{z}
  g(\kappa z)
  -
  \left(
    \kappa^{2}
    z^{2}
    +
    \alpha^{2}
  \right)
  g(\kappa z)
  &=
  0
\end{align*}
where $\alpha \in \mathbb{R}$. Choosing $\alpha = 0$ and dividing by $z^{2}$, which is no problem, as we want the equation to hold only away from $0$, we obtain exactly the eigenvalue equation. Thus, the solutions for the modified Bessel equation with $\alpha = 0$ also solve the eigenvalue equation, as $f = f(\Vert \cdot \Vert)$. Now there are two linearly independent solutions $I_{\alpha}$ and $K_{\alpha}$ to Bessel's modified equation, but only $K_{0}$ diverges of logarithmic type as $z \downarrow 0$. So $K_{0}$ is the desired function. For this Bessel function we can write in the limit of small arguments (see e.g. \cite{69346099})
\begin{align*}
  K_{0}(z)
  &=
  -
  \gamma
  +
  \ln(2)
  -
  \ln(z)
  +
  o(z)
  \qquad
  (z \downarrow 0)
  \,,
\end{align*}
where $\gamma \approx 0.577$ is the Euler-Mascheroni constant. We thus include the functions $Af_{\kappa,y}(x) := AK_{0}(\kappa\Vert x - y \Vert)$, where $A \in \mathbb{C}$ and $\kappa > 0$, in the domain of the one-particle Hamilton operator. For these we have
\begin{align*}
  -
  \lim_{r \downarrow 0}
  \int_{S^{1}}
  r
  \partial_{r}
  A
  f_{\kappa,y}(y + r\varphi)
  \mathrm{d}\varphi
  &=
  -
  \lim_{r \downarrow 0}
  \int_{S^{1}}
  r
  \partial_{r}
  A
  K_{0}(\kappa r)
  \mathrm{d}\varphi
  \\
  &=
  -
  2\pi
  A
  \lim_{r \downarrow 0}
  r
  \left(
    -
    \frac{1}{r}
    +
    o(1)
  \right)
  =
  2\pi
  A
\end{align*}
Now note that
\begin{align*}
  \lim_{r \downarrow 0}
  \frac{1}{\ln(r)}
  A
  f_{\kappa,y}(y + r\varphi)
  &=
  -
  A
  \lim_{r \downarrow 0}
  \frac{\ln(\kappa r)}{\ln(r)}
  =
  -
  A
\end{align*}
and that for continuous $\psi^{n}$ we have
\begin{align*}
  \lim_{r \downarrow 0}
  \frac{1}{\ln(r)}
  \psi^{n}
  \left(
    x_{1}
    ,
    \dots
    ,
    x_{n-1}
    ,
    y
    +
    r\varphi
  \right)
  &=
  0
  \,.
\end{align*}
Altogether the equality
\begin{align*}
  -
  \lim_{r \downarrow 0}
  \int_{S^{1}}
  r
  \partial_{r}
  \psi^{n}
  \left(
    x_{1}
    ,
    \dots
    ,
    x_{n-1}
    ,
    y
    +
    r
    \varphi
  \right)
  \mathrm{d}\varphi
  &=
  -
  2\pi
  \lim_{r \downarrow 0}
  \frac{1}{\ln(r)}
  \psi^{n}
  \left(
    x_{1}
    ,
    \dots
    ,
    x_{n-1}
    ,
    y
    +
    r\varphi
  \right)
\end{align*}
holds for the functions considered\footnote{the sufficiently regular functions and the $K_{0}$-functions}. We thus define an operator $\mathcal{B}$ on an appropriate dense domain $D_{\mathcal{B}} \subset \mathcal {F}$ sector-wise by
\begin{align*}
  (\mathcal{B}\psi)^{n}
  \left(
    x_{1}
    ,
    \dots
    ,
    x_{n}
  \right)
  &:=
  2\pi
  \sqrt{n + 1}
  \lim_{r \downarrow 0}
  \frac{1}{\ln(r)}
  \psi^{n + 1}
  \left(
    x_{1}
    ,
    \dots
    ,
    x_{n}
    ,
    y
    +
    r\varphi
  \right)
  \,.
\end{align*}
Looking back at equation \eqref{ibcbase} we can now state our IBC as $\mathcal{B}\psi = c\psi$.
\\
Now we have a problem with the definition of the annihilation operator since the functions $f_{\kappa,y}$ cannot be evaluated at $y$. To find a workaround for this note that
\begin{align*}
  \partial_{z}
  \frac{1}{\ln(z)}
  &=
  -
  \frac{1}{z\ln^{2}(z)}
  \,.
\end{align*}
Using Lebesgue's dominated convergence theorem one hereby easily concludes that for continuously differentiable $\psi^{n}$,
\begin{align*}
  \psi^{n}
  \left(
    x_{1}
    ,
    \dots
    ,
    x_{n - 1}
    ,
    y
  \right)
  &=
  -
  \frac{1}{2\pi}
  \lim_{r \downarrow 0}
  r\ln^{2}(r)
  \partial_{r}
  \frac{1}{\ln(r)}
  \int_{S^{1}}
  \psi^{n}
  \left(
    x_{1}
    ,
    \dots
    ,
    x_{n - 1}
    ,
    y
    +
    r
    \varphi
  \right)
  \mathrm{d}\varphi
  \,.
\end{align*}
The right-hand side of this equation is also well-defined for the functions that diverge like $f_{\kappa,y}$, which is a consequence of
\begin{align*}
  \lim_{z \downarrow 0}
  \partial_{z}
  \frac{z}{\ln(z)}
  &=
  \lim_{z \downarrow 0}
  \left(
    \frac{1}{\ln(z)}
    -
    \frac{1}{\ln^{2}(z)}
  \right)
  =
  0
  \,.
\end{align*}
Therefore, an operator $\mathcal{A}$ that captures the meaning of the annihilaton operator $a(\delta_{y})$ can be defined, on an appropriate dense domain $D_{\mathcal{A}} \subset \mathcal{F}$, sector-wise by
\begin{align*}
  (\mathcal{A}\psi)^{n}
  \left(
    x_{1}
    ,
    \dots
    ,
    x_{n}
  \right)
  &:=
  -
  \frac{\sqrt{n + 1}}{2\pi}
  \lim_{r \downarrow 0}
  r\ln^{2}(r)
  \partial_{r}
  \frac{1}{\ln(r)}
  \int_{S^{1}}
  \psi^{n + 1}
  \left(
    x_{1}
    ,
    \dots
    ,
    x_{n}
    ,
    y
    +
    r\varphi
  \right)
  \mathrm{d}\varphi
  \,.
\end{align*}
Now we have a model for our two-dimensional system given by
\begin{align*}
  H
  =
  -
  \Delta^{\mathcal{F}}
  +
  \overline{c}
  \mathcal{A}
  \qquad
  &\text{with IBC}
  \qquad
  \mathcal{B}\psi
  =
  c
  \psi
  \,.
\end{align*}
As mentioned in the introduction the domain $D_{\textrm{IBC}}$ of $H$ is not that easy to find and even harder to handle, e.g. showing its denseness. We will only sketch here how it is given in \cite{7f9f7f95}, where the three-dimensional case is considered. We suppose that all steps can be transferred to the two-dimensional case treated here and thus replace $\mathbb{R}^{3}$ by $\mathbb{R}^{2}$ at every occurrence. Let $\Delta^{(n)} := \sum_{j = 1}^{n}\Delta_{j}$ be the Laplace operator in the $n$-th sector and let $\mathcal{C}^{n} \subset \mathbb{R}^{2n}$ denote the set of those $n$-particle configurations with at least one particle at the location of the source. The authors choose the set
\begin{align*}
  D(\Delta^{(n)})
  :=
  H_{0}^{2}
  \left(
    \mathbb{R}^{2n}
    \setminus
    \mathcal{C}^{n}
  \right)
  \subset
  L^{2}(\mathbb{R}^{2n})
\end{align*}
as a domain for $\Delta^{(n)}$, where this set is defined as the closure of the smooth and compactly supported functions $C_{0}^{\infty}(\mathbb{R}^{2n} \setminus \mathcal{C}^{n})$ with respect to the $H^{2}$-norm. They consider the adjoint of this operator, $(\Delta^{(n)\ast},D(\Delta^{(n)\ast}))$, which extends $\Delta^{(n)}$ since the latter is densely defined, closed and symmetric. The domain of the adjoint is given by
\begin{align*}
  D(\Delta^{(n)\ast})
  &=
  D(\Delta^{(n)})
  \oplus
  \mathrm{ker}
  \left(
    \Delta^{(n)\ast}
    -
    \mathrm{i}
  \right)
  \oplus
  \mathrm{ker}
  \left(
    \Delta^{(n)\ast}
    +
    \mathrm{i}
  \right)
  \,.
\end{align*}
In the definition of the Hamilton operator $H$ the adjoint is then used instead of the Laplace operator and the domain of $H$ is defined as
\begin{align*}
  D_{\textrm{IBC}}
  &:=
  \left\lbrace
    \psi
    \in
    \mathcal{F}
    \colon
    \psi^{n}
    \in
    D(\Delta^{(n)\ast})
    \cap
    \mathcal{H}^{n}
    \text{ for all }
    n
    \in
    \mathbb{N}^{\times}
    ,
    H
    \psi
    \in
    \mathcal{F}
    ,
    \mathcal{A}
    \psi
    \in
    \mathcal{F}
    \text{ and }
    \mathcal{B}
    \psi
    =
    c
    \psi
  \right\rbrace
  \,.
\end{align*}
This necessarily requires that $D_{\textrm{IBC}} \subset D_{\mathcal{A}}$ and $D_{\textrm{IBC}} \subset D_{\mathcal{B}}$.
